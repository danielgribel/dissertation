% -*- coding: utf-8; -*-
\chapter{Problem Statement and Literature Review}

\section{Ad-hoc methods}

\subsection{Hierarchical clustering}

\subsection{Partitional clustering}

\section{Model-driven clustering}

\section{The MSSC problem}
In the Minimum sum-of-squares clustering (MSSC) problem we aim to minimize the total sum of the distances of each data point to the mean point in its cluster. It considers as an assumption a given set $X$ of $n$ data points in a $d$-dimensional space $\mathbb{R}^d$ \cite{Bagirov2006}:

\begin{center}%
$X = \{x_1, x_2, ..., x_n\}$, where $x_i \in \mathbb{R}^d, \quad i = 1, ..., n.$
\end{center}

Then, if we consider $S$ as the set of all possible subsets obtained from elements in $X$, the clustering procedure must generate a partition $\bar{S}$ of $S$ with $m$ subsets $S_k \in S (S_k \neq \emptyset)$, where $m$ is the number of desired clusters and $k = \{1,...,\left | S \right |\}$. In other words, a clustering algorithm aims to generate a partition that groups data consistently, with most similar objects belonging to the same group and dissimilar objects belonging to different groups.

In order to define the similarity between two objects, a function $d$ assigns to each pair $(x_i, x_j)$ a distance or similarity metric $d_{ij} \in \mathbb{R}$, where $d_{ij} \geq 0, d_{ii} = 0, d_{ij} = d_{ji}$, for $i, j = 1, ... ,n$. The set of objects $x_i \in X$ can be described as a matrix $x_{iq}$, with $q = \{1,...,d\}$, where $x_{iq}$ is the value of the $q$-th feature for the $i$-th data object and $d$ is the number of features. The assignment is done in such a way that the greater is the similarity (proximity) within a cluster and the greater is the difference between clusters, the better is the clustering. There is a variety of distance measures to characterize how similar are two objects or patterns. A popular one in the domain of data mining is the Euclidean distance, that can often be used to reflect the similarity between two patterns when considering multiple dimensions (features) \cite{Jain1999}:

\begin{equation}
d_{ij} = \sqrt{\sum_{q=1}^{d}(x_{iq} - x_{jq})^2} = \left \| x_i - x_j \right \|
\end{equation}

Thus, given a distance matrix $d_{ij}$, the MSSC problem can be formulated as the following set partitioning problem:

\begin{equation} \label{eq:of}
\textrm{Minimize} \sum_{k=1}^{\left | S \right |}c_ky_k
\end{equation}

%\begin{equation}
%\sum_{i=1}^{N}\sum_{k=1}^{\left | S \right |}a_{ik}y_k = 1
%\end{equation}

\begin{equation}
\sum_{k=1}^{\left | S \right |}a_{ik}y_k = 1, \quad \forall i
\end{equation}

\begin{equation}
\sum_{k=1}^{\left | S \right |}y_k = m
\end{equation}

\begin{equation}
a_{ik} \in \{0,1\}, \quad \forall i,k
\end{equation}

\begin{equation}
y_k \in \{0,1\}, \quad \forall k
\end{equation}

\noindent where $\left | S \right |$ is the size of the set $S$; $a_{ik} = 1$ if $x_i \in S_k$ and $a_{ik} = 0$ otherwise; $y_k$ are the decision variables, with $y_k = 1$ if the subset $S_k$ is chosen ($S_k \in \bar{S}$) and $y_k = 0$ otherwise; and $c_k$ is the cost function on subset $S_k$, i.e., the cost within the $k$-th subset. In MSSC problem, the centroid is the mean point $\mu_k$ of cluster $S_k$. Thus, the contribution $c_k$ of each subset to the objective function is:
	\begin{equation}
	c_k = \sum_{x_i \in S_k}\left \| \mu_k - x_i \right \|
	\end{equation}
	\begin{center}
	where $\mu_k = $ \Large $\frac{\sum_{i = 1}^{n}a_{ik}x_i}{\sum_{i = 1}^{n}a_{ik}}$	
	\end{center}
	
\subsection{Computational complexity}

\section{Solution techniques}

\subsection{Construction and local search}

\subsection{Meta-heuristics}

\subsection{Mathematical programming methods}