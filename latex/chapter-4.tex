\chapter{Methodology}
\label{chap:methodology}
This chapter describes the proposed method to solve the MS-Mean problem, the most treated objective in data clustering. The designed method is based on a genetic algorithm (GA) with local improvements combined with mechanisms that allow the diversification (local minima escape) and the propagation of good solutions. In this method, the local improvement is done through the running of the k-means algorithm, which takes one candidate solution in the GA population as a starting point. In other words, the method may be defined as a multi-start k-means inside a GA framework, which in turn is guided by the MS-Mean objective.

\section{General structure}
The general scheme of the meta-heuristic here addressed is described by algorithm \ref{genetic-algo}. Initially, the method generates a random population of individuals. Then, it applies successively a number of operators to evolve this population: firstly, it selects two parent individuals from the population and combines them by a crossover procedure, yielding to a new individual (offspring) that is added to the population. Secondly, the offspring is mutated and enhanced by a local search, generating a new individual solution that is also added to the population. These two steps -- crossover and mutation with enhancement -- are performed many times until a termination criteria is reached. Additionally, two mechanisms -- one to select survivor individuals and the other to diversify the population -- are applied when some criteria are reached. These mechanisms allow the method to keep and propagate the best individuals and to introduce some diversification, respectively. The following sub-sections describes how these operators and mechanisms to evolve and manage the population of solutions work.

\section{Solution representation}
A solution (clustering partition) is represented by a direct encoding of the object-cluster assignment. The idea is to use a genetic encoding that allocates directly $n$ objects to $k$ clusters, such that each candidate solution consists of a $n$-vector ($n$ genes) with integer values in the interval [1, $k$]. Thus, for $n$ = 4 and $k$ = 3, the encoding (1,1,3,2) allocates the first and the second objects to cluster 1; the third object to cluster 3 and the fourth object to cluster 2, generating the partition (\{1,2\}, \{3\}, \{4\}).

An encoding based on the centroids description is also considered in order to make the proposed crossover and mutation operations practicable. A centroids encoding describes a solution through the feature vectors existing in each centroid, being a matrix $C$ of size $k \times d$, where $k$ is the number of centroids (clusters), $d$ is the number of features and an entry $c_{ij}$ of $C$ corresponds to the value of the $j$-th feature of the $i$-th centroid ($i$ = 1 ... $k$; $j$ = 1 ... $d$).

%An encoding based on the centroids description is also considered in order to treat the redundancy that may arise from the object-cluster representation. For instance, the solutions (1,1,3,2) and (3,3,1,2) may be considered different solutions according to the object-cluster assignment, even though they are exactly the same solution. Thus, a centroids encoding is used to represent solutions whenever this redundancy must be detected, for example, when performing the crossover.

\begin{algorithm}[H]
\caption{Genetic algorithm framework}
\label{genetic-algo}
\begin{algorithmic}[1]
\STATE Initialize population
\WHILE{number of iterations without improvement $< itNoImprovement$}
\STATE Select parents $p_1$ and $p_2$
\STATE Generate an offspring $\theta$ from $p_1$ and $p_2$ (crossover)
\STATE Generate an individual $\theta'$ by mutating $\theta$ (mutation)
\STATE Apply local search on $\theta'$
\STATE Add $\theta$ and $\theta'$ to population
\IF{population size = $maxPop$}
\STATE Select survivors
\IF{best solution not improved for $itDiv$ iterations}
\STATE Diversify population
\ENDIF
\ENDIF
\ENDWHILE
\STATE Return best solution
\end{algorithmic}
\end{algorithm}

\section{Initial population}
The initial population is generated randomly, by assigning each object to a cluster according to a discrete uniform distribution, i.e., where each outcome is equally likely to happen. To compose the initial population, $\mu$ individuals are created. Then, each initial individual undergo a local search.

\section{Individuals management}
The generation of a new individual (offspring) begins with the selection of two parents, $p_1$ and $p_2$, which are submitted to a crossover procedure that generates a single individual $\theta$. Then, a local search is applied to $\theta$, which is finally added to the population.

\subsection{Selection}
The parent selection is done through a binary tournament, which randomly selects $w$ individuals (with uniform probability) from the population and keeps the one with the best fitness among the $w$ individuals to set the first parent. The fitness here considered is the value of the objective function (cost) of a solution. Then, the same selection scheme is performed to set the second parent.

\subsection{Crossover}
Once parent solutions $p_1$ and $p_2$ were selected, they should be submitted to a crossover procedure in order to produce a offspring (child) solution from them:

\begin{enumerate}
	\item Firstly, the minimum bipartite matching between centroids of $p_1$ and $p_2$ is calculated. One set of nodes in the matching problem is composed by the centroids of $p_1$ and the other set is composed by the centroids of $p_2$. The goal is to produce the one-to-one assignment of centroids from different sets, in such a way that the overall edges weight (distances) is minimized. This process is solved by the Hungarian method and leads to $k$ pairs of centroids, where $k$ is the number of clusters (centroids).

	\item For each pair of centroids, one of them is randomly selected and set as a centroid of the offspring solution.

	\item Finally, data points are assigned to the closest offspring centroid.
\end{enumerate}

\subsection{Mutation}
The mutation in a solution is done according to the following steps:

\begin{enumerate}

	\item Randomly select one centroid and remove it from the solution.
	
	\item Among the $m-1$ remaining centroids, re-assign data points to the closest one.
	
	\item Randomly select the position of a data point and re-insert the removed centroid there.
	
	\item Among the $m$ centroids, re-assign data points to the closest one.
	
\end{enumerate}

\subsection{Local improvement}
Once an offspring $\theta$ was generated by crossover, it is improved through a local search procedure. The local search aims to find a local optimal by applying local changes to the current solution. Here, the adopted local search is one run of the k-means algorithm. The k-means algorithm starts with an initial solution with $k$ centroids $c_1$, $c_2$, ..., $c_k$, and proceeds by alternating between two steps:

\begin{itemize}

	\item Assignment step: Assign each data point to the closest cluster.

	\item Update step: Calculate the new centroids to be the mean (average) point of the data points in the new clusters.
	
\end{itemize}

Then the algorithm keeps repeating these two steps until the assignments no longer change, converging to a local optima.

We used the k-means implementation of \cite{Hamerly2010}, who proposed an acceleration that gives the same answer of the standard Lloyd's k-means but is much faster in practice. This implementation avoids distance computations by using the triangle inequality and lower bounds on distances. The time per k-means iteration of the algorithm is O($ndk$+$dk^2$), where $n$ is the number of data points, $d$ is the number of data dimensions (features) and $k$ is the number of clusters. However, the calculated lower bounds allow to eliminate the innermost k-means loop in around 80\% of the time, which in practise is much faster than the standard k-means.

\section{Population management and termination}
Two mechanisms were developed in order to complement the selection, crossover and local improvement operators: Survivors selection and Diversification management.

\subsection{Survivors selection}
The \textit{Survivors selection} aims to select the best individuals to propagate when the maximum population size is achieved. This procedure determines the $\mu$ individuals that will go on to the next generation, by discarding $\lambda$ individuals ($\lambda = maxPop - \mu$) that are either clones or bad regarding the fitness, according to algorithm \ref{survivors}.

\begin{algorithm}[H]
\caption{Survivors selection}
\label{survivors}
\begin{algorithmic}[1]
\FOR{$i = 1 ... \lambda$}
\STATE $X \leftarrow $ all individuals having a clone
\IF{$X \neq \emptyset$}
\STATE Remove $p \in X$ with maximum fitness
\ELSE
\STATE Remove $p$ in the population with maximum fitness
\ENDIF
\ENDFOR
\end{algorithmic}
\end{algorithm}

\subsection{Diversification management}
The \textit{Diversification} mechanism aims to ensure the diversity of the population. It is called after the survivors selection whenever \textit{itDiv} iterations happen without improving the best solution. It is performed by creating $\beta$ new individuals as in the initialization phase, i.e., individuals are randomly generated and then submitted to local search.

\section{Parameters}
In preliminary experiments, we determined that the values $\mu$ = 20, $maxPop$ = 500, $itDiv$ = 400 and $itNoImprovement$ = 2000 produced good and stable results, where $\mu$ is the population size, $maxPop$ is the maximum size of population, $itDiv$ is the number of iterations without improvement that triggers diversification and $itNoImprovement$ is the number of iterations without improvement that causes the algorithm stop.