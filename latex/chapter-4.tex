\chapter{Computational Experiments and Analysis}

\section{Instances}
The selection of instances was done in order to cover different types of data, considering both the instance size and the number of features. Table \ref{instances} summarizes the characteristics of these instances, which are important benchmarks in recent clustering literature, as reported in \cite{Ordin2014} and \cite{Bagirov2016}.

\begin{table}[]
\centering
\begin{tabular}{@{}lcc@{}}
\toprule
Instance                        & Number of data points & Number of features \\ \midrule
German towns                    & 59                    & 2                  \\
Bavaria postal 1                & 89                    & 3                  \\
Bavaria postal 2                & 89                    & 4                  \\
Fisher’s Iris Plant             & 150                   & 4                  \\
Heart Disease                   & 297                   & 13                 \\
Liver Disorders                 & 345                   & 6                  \\
Ionosphere                      & 351                   & 34                 \\
Congressional Voting Records    & 435                   & 16                 \\
Breast Cancer                   & 683                   & 9                  \\
Pima Indians Diabetes           & 768                   & 8                  \\
TSPLIB1060                      & 1060                  & 2                  \\
Image Segmentation              & 2310                  & 19                 \\
TSPLIB3038                      & 3038                  & 2                  \\
Page Blocks                     & 5473                  & 10                 \\
Pendigit                        & 10992                 & 16                 \\
Gas sensor                      & 13910                 & 128                \\
EEG eye state                   & 14980                 & 14                 \\
D15112                          & 15112                 & 2                  \\
Letters                         & 20000                 & 16                 \\
KEGG metabolic relation network & 53413                 & 20                 \\
Shuttle control                 & 58000                 & 9                  \\
Pla85900                        & 85900                 & 2                  \\
Skin Segmentation               & 245057                & 3                  \\
3D road network                 & 434874                & 3                  \\ \bottomrule
\end{tabular}
\caption{Instances description}
\label{instances}
\end{table}

\section{Parameters calibration}
As in most meta-heuristics, the values chosen for parameters directly affect the quality of the results. In this work, we consider five main parameters: $\mu$ (the population size), $\eta$ (the maximum size of population), $\beta$ (the number of new individuals created in the diversification), $I$ (the number of iterations without improvement that triggers diversification) and $itNoImprovement$ (the number of iterations without improvement that causes the algorithm stop).

In preliminary experiments, we started with the following configuration, as they produced good and stable results: $\mu$ = 30, $\eta$ = 400, $\beta$ = 5, $I$ = 400 and $itNoImprovement$ = 2000. From this baseline, we expanded the range of these value. Table \ref{calibration} presents the tested values for each parameter and the final values achieved after running each combination in XX instances.

\begin{table}[H]
\centering
\begin{tabular}{@{}llcc@{}}
\toprule
\multicolumn{2}{l}{Parameter}                                                                                                    & Tested values   & Final value \\ \midrule
$\mu$         & Population size                                                                                                  & \{20, 30, 50\}    & 30               \\
$\beta$       & Number of new individuals in \textit{Diversification}                                                                     & \{5, 10, 20\}     & 5                \\
$\eta$ & Maximum size of population                                                                                       & \{200, 400, 800\} & 400              \\
$I$         & \begin{tabular}[c]{@{}l@{}}Number of iterations without improvement\\ that triggers \textit{Diversification}\end{tabular} & \{200, 400, 800\} & 400              \\ \bottomrule
\end{tabular}
\caption{Parameters calibration}
\label{calibration}
\end{table}

\section{Results}